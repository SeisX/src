\author{Team: Longhorns}
%%%%%%%%%%%%%%%%%%%%%%%%
\title{GEO 365N/384S Seismic Data Processing \\ Computational Assignment 6}

\begin{abstract}
In this assignment, you will experiment with different kinds of depth migration:
  \begin{enumerate}
  \item Post-stack phase-shift (Gazdag) migration and its approximation by Stolt stretch.
  \item Prestack Kirchhoff migration.
  \item Post-stack reverse-time migration (RTM).
  \end{enumerate}
Migration will be applied to previously processed datasets: Viking Graben and Teapot Dome data.
\end{abstract}

\section{Prerequisites}

Completing the computational part of this homework assignment requires
\begin{itemize}
\item \texttt{Madagascar} software environment available from \\
\url{http://www.ahay.org/}
\item \LaTeX\ environment with \texttt{SEG}\TeX\ available from \\ 
\url{http://www.ahay.org/wiki/SEGTeX}
\end{itemize}
To do the assignment on your personal computer, you need to install
the required environments. 

To setup the Madagascar environment in the JGB 3.216B computer lab, run the following commands:
\begin{verbatim}
module load madagascar-devel
source $RSFROOT/share/madagascar/etc/env.csh
setenv DATAPATH $HOME/data/
setenv RSFBOOK $HOME/data/book
setenv RSFFIGS $HOME/data/figs
\end{verbatim}
You can put these commands in your \verb+$HOME/.cshrc+ file to run them automatically at login time.

To setup the \LaTeX\ environment, run the following commands:
\begin{verbatim}
cd
git clone https://github.com/SEGTeX/texmf.git
texhash
\end{verbatim}
You only need to do it once.

The homework code is available from the class repository
by running
\begin{verbatim}
svn co https://github.com/GEO384S/geo384s/trunk/hw6
\end{verbatim}
You can also download it from your team's private repository.

\section{Generating this document}

At any point of doing this computational assignment, you can
regenerate this document and display it on your screen.

\begin{enumerate}          
\item Change directory to \texttt{hw6}:
\begin{verbatim}
cd hw6
\end{verbatim}
\item Run
\begin{verbatim}
sftour scons lock
scons read &
\end{verbatim}
\end{enumerate}

As the first step, open \texttt{hw6/paper.tex} file in your favorite
editor and edit the first line to enter the name of your team. Then
run \texttt{scons read} again.

\section{Phase-shift migration}
\inputdir{gazdag}

In the first part of the assignment, we will return to processing 3D land data, the Teapot Dome dataset.

\begin{enumerate}

\item Change directory to \texttt{hw6/gazdag}.
\item Run
\begin{verbatim}
scons -c
\end{verbatim}
to remove (clean) previously generated files.

\item To simplify the processing, we will use a single velocity function assuming a laterally homogeneous medium. This velocity can be picked from a supergather combining traces from different locations. Figure~\ref{fig:vscan} shows the semblance scan using every 500th trace in the data and the corresponding picked NMO velocity trend. To reproduce it on your screen, run
\begin{verbatim}
scons vscan.view
\end{verbatim}

\plot{vscan}{width=0.8\textwidth}{Semblance scan of a supergather (every 500th trace) from the Teapot Dome dataset. The curve shows the picked velocity trend.}

\sideplot{velocity}{width=\textwidth}{Interval velocity in the Teapot Dome dataset estimated by Dix inversion (regularized by smoothing). The dashed curve shows the corresponding picked RMS velocity.}

\item In the next step, we will treat the NMO velocity as the RMS velocity and convert it to interval velocity using Dix inversion \cite[]{GEO20-01-00680086}. To display the result (Figure~\ref{fig:velocity}), run
\begin{verbatim}
scons velocity.view
\end{verbatim}
To avoid instabilities, Dix inversion is assisted by smoothing regularization.
\item Using the picked NMO velocity, we can apply NMO stacking to create a 3D stacked cube (Figure~\ref{fig:stack}). To display it on your screen, run
\begin{verbatim}
scons stack.view
\end{verbatim}

\plot{stack}{width=0.8\textwidth}{Stack of the Teapot Dome dataset.}

\item The stack in Figure~\ref{fig:stack} covers a region with an irregular shape. For further processing, we will select a portion of the data and rotate it to align with the axes (Figure~\ref{fig:stack2}). To perform windowing and rotation, run
\begin{verbatim}
scons stack2.view
\end{verbatim}

\item A depth migration method designed specifically for laterally-invariant $V(z)$ velocity distributions is phase-shift migration, also known as Gazdag migration \cite[]{GEO43-07-13421351}. The phase-shift method extrapolates the recorded wavefield in depth.

Applying the Fourier transform in time and in lateral spatial directions to the wave equation turns it into an
ordinary differential equation
\begin{equation}
\label{eq:kwave}
\frac{d^2 \hat{U}}{d z^2} + \lambda^2\,\hat{U} = 0\;,
\end{equation}
where $U(\omega,\mathbf{k},z)$ is the wavefield corresponding to temporal frequency $\omega$, lateral wavenumber $\mathbf{k}$, and depth $z$, and
\[
\lambda^2 = \frac{\omega^2}{V^2(z)} - \mathbf{k} \cdot
\mathbf{k}
\]
Equation~(\ref{eq:kwave}) admits a depth-stepping solution
\begin{equation}
\label{eq:ps}
\hat{U}(z+\Delta z,\mathbf{k},\omega) = \hat{U}(z,\mathbf{k},\omega)\,e^{\pm i\,\lambda\,\Delta z}\;,
\end{equation}
which is the basis of the phase-shift method.

To generate a 3D phase-shift migration of the stack in Figure~\ref{fig:stack}, run
\begin{verbatim}
scons image.view
\end{verbatim}
The computation proceeds in the vertical time coordinates, with the result displayed on the same grid as the input. To compare the data before and after migration, run
\begin{verbatim}
sfpen Fig/stack2.vpl Fig/image.vpl
\end{verbatim}
Do you observe notable differences?

\multiplot{2}{stack2,image}{width=0.8\textwidth}{Portion of the rotated Teapot Dome stack (a) and its migration by the phase-shift method displayed in vertical time (b).}

\item Rewrite equations~(\ref{eq:kwave}) and~(\ref{eq:ps}) using the vertical time coordinate 
\[
\tau = \int\limits_{0}^{z} \frac{2\,d\zeta}{V(\zeta)}
\]
instead of $z$. 

\answer{
}

Which sign (plus or minus) should be used in
equation~(\ref{eq:ps}) for post-stack migration?

\multiplot{2}{mig10,migst}{width=0.8\textwidth}{Portion of the rotated Teapot Dome stack migrated using Stolt migration. (a) Using a constant velocity of 10 kft/s. (b) Using the Stolt stretch method.}

\item In the classic paper, \cite{GEO50-11-22192244} proposed not only a Fourier-domain method for a constant-velocity migration but also an effective method for approximating the output of phase-shift migration in a $V(z)$ medium. The method became known as \emph{Stolt stretch}. It consists of three steps:
\begin{enumerate}
\item Stretching the data in time by transforming from time $t$ to stretched time $t_s$ according to
\begin{equation}
t_s(t)={\left(\frac{2}{V_0^2}\,\int\limits_0^t \tau\,V_r(\tau) d \tau\right)}^{1/2}\;,
\label{eqn:ss} 
\end{equation}
where $V_0$ is a reference constant velocity, and $V_r(t)$ is the RMS velocity:
\[ 
V_r(t) = \frac{1}{t} \int\limits_0^t V^2(\tau) d \tau\;.
\]
\item Performing Stolt migration according to mapping
\begin{equation}
\omega = 
\left(1-\frac{1}{W}\right) \omega_0+
\frac{\omega_0}{W}\,
\sqrt{1 + W\,\frac{V_0^2 k^2}{4\,\omega_0^2}}\;,
\label{eqn:sdispersion} 
\end{equation}
where $W$ is a constant (typically between 1 and 2).

\item Inverse stretch (squeeze) from $t_s$ to $t$.
\end{enumerate} 

Figure~\ref{fig:mig10} shows the result of Stolt migration without stretch and using velocity of 10 kft/s. Figure~\ref{fig:migst} is the result of Stolt migration using the stretch approach. To display these figures on your screen, run
\begin{verbatim}
scons mig10.view migst.view
\end{verbatim}

\item How much faster is Stolt migration compared to Gazdag migration? 

You can compare the CPU time of different methods experimentally by
running \texttt{scons TIMER=y} instead of \texttt{scons}.

\answer{
}

\item How accurate is Stolt stretch in approximating the output of Gazdag migration? You can compare the results by running
\begin{verbatim}
sfpen Fig/image.vpl Fig/migst.vpl
\end{verbatim}

Try improving the match by modifying the value of the Stolt-stretch parameter $W$ from the value of $W=1.5$. You can find a better value by experimentation or by using the appropriate theory \cite[]{stoltst}.

\item (\textbf{EXTRA CREDIT}) For an extra credit, try improving the imaging result by using a more detailed velocity analysis or a more accurate migration method.

\end{enumerate}

\lstset{language=python,numbers=left,numberstyle=\tiny,showstringspaces=false}
\lstinputlisting[frame=single,title=gazdag/SConstruct]{gazdag/SConstruct}

\section{Reverse-time migration}
\inputdir{rtm}

In this part of the assignment, we will create a depth image of the
Viking Graben data using post-stack reverse-time migration (RTM).

\begin{enumerate}

\item Change directory to \texttt{hw6/rtm}.
\item Run
\begin{verbatim}
scons -c
\end{verbatim}
to remove (clean) previously generated files.
\item We start by reproducing the DMO processing workflow from Assignment 3 using the recipe from \emph{Radii Interceptor}. A picked DMO (time-migration) velocity distribution is shown in Figure~\ref{fig:vdix}. To reproduce it on your screen, run
\begin{verbatim}
scons vdix.view
\end{verbatim}
\item The next step is converting the time-migration velocity to depth for creating an initial velocity model for depth migration.  To display the Dix-converted velocity in time and depth (Figure~\ref{fig:vdix,vofz}), run
\begin{verbatim}
scons vdix.view vofz.view
\end{verbatim}
\item The DMO stack (Figure~\ref{fig:slice}) can be used as the input to post-stack depth migration. We will use reverse-time migration by the lowrank approximation method \cite[]{lowrank}. The method employs several (in this case, three) spatial Fourier transforms per time step. To generate the result (Figure~\ref{fig:rtm}), run
\begin{verbatim}
scons rtm.view
\end{verbatim}
To watch a movie of reverse-time wave propagation which produced the image in Figure~\ref{fig:rtm}, run
\begin{verbatim}
scons snaps.vpl
\end{verbatim}
\item Your task: try to improve the quality of the image by 
\begin{enumerate}
\item using the results from your own time-domain imaging (Assignments 1--5);
\item modifying the time-to-depth conversion step.
\end{enumerate}

\plot{vpick}{width=0.8\textwidth}{Picked DMO (time-migration) velocity in the Viking Graben dataset.}

\multiplot{2}{vdix,vofz}{width=0.8\textwidth}{Interval velocity estimated by Dix inversion in vertical time (a) and depth (b).}

\multiplot{2}{slice,rtm}{width=0.8\textwidth}{DMO stack of the Viking Graben dataset (a) and its depth migration by RTM (b).}

\end{enumerate}

\lstset{language=python,numbers=left,numberstyle=\tiny,showstringspaces=false}
\lstinputlisting[frame=single,title=rtm/SConstruct]{rtm/SConstruct}

\section{Kirchhoff migration}
\inputdir{kirchhoff}

How accurate is the image in Figure~\ref{fig:rtm}? We can evaluate it
by performing prestack depth migration (PSDM). In this part of the
assignment, we will image the same dataset using prestack Kirchhoff
depth migration method.

\plot{shots}{width=0.8\textwidth}{Preprocessed shot gathers from the Teapot Dome dataset.}

\plot{psdm}{width=0.8\textwidth}{Kirchoff prestack depth migration of the Teapot Dome dataset (using every 10th shot).}

\begin{enumerate}

\item Change directory to \texttt{hw6/kirchhoff}.
\item Run
\begin{verbatim}
scons -c
\end{verbatim}
to remove (clean) previously generated files.
\item For the prestack migration exercise, we will use the same input as in the previous section but resort it into shot gathers (Figure~\ref{fig:shots}). To display the data on your screen, run
\begin{verbatim}
scons shots.view
\end{verbatim}
\item The first step in the Kirchhoff method is computing traveltime tables. To display the computed table on your screen (as a movie over source locations), run
\begin{verbatim}
scons times.vpl
\end{verbatim}
In addition to traveltimes themselves, we compute the traveltime
derivatives with respect to the source and receiver
locations \cite[]{eikods}.
\item To compute and display the PSDM image (Figure~\ref{fig:psdm}), run 
\begin{verbatim}
scons psdm.view
\end{verbatim}
Caution: this computation is expensive and may take some time.

\item Your task: replace input data with the result of your processing (parabolic Radon demultiple) from the Computational Assignment 5 and compare the results.

\item (\textbf{EXTRA CREDIT}) For an extra credit, replace \texttt{cig=n} with \texttt{cig=y} in the \texttt{SConstruct} file to generate CIGs (Common Image Point Gathers) instead of a stacked migration image. Examine the flatness of the gathers as an indication of the velocity model correctness. Can you improve the image by updating the velocity model or processing the gathers?

\end{enumerate}

\lstset{language=python,numbers=left,numberstyle=\tiny,showstringspaces=false}
\lstinputlisting[frame=single,title=kirchhoff/SConstruct]{kirchhoff/SConstruct}

%\section{Saving your work}

%It is important to save all files that you edit by hand (such
%as \texttt{paper.tex} and \texttt{SConstruct}) in a version-control
%system every time you make a revision.  The completed assignment is
%due in two weeks and should be submitted through your team's private
%GitHub repository.

\bibliographystyle{seg}
\bibliography{SEG,hw6}


